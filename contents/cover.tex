\cqusetup{
%	************	注意	************
%	* 1. \cqusetup{}中不能出现全空的行,如果需要全空行请在行首注释
%	* 2. 不需要的配置信息可以放心地坐视不理、留空、删除或注释(都不会有影响)
%	*
%	********************************
% ===================
%	论文的中英文题目
% ===================
  ctitle = {面向车载边缘计算的信息物理融合系统\\关键技术研究},
  etitle = {Research on Key Techniques for Cyber-Physical System in Vehicular Edge Computing},
% ===================
% 作者部分的信息
% \secretize{}为盲审标记点,在打开盲审开关时内容会自动被替换为***输出,盲审开关默认关闭
% ===================
  cauthor = \secretize{许新操},	% 你的姓名,以下每项都以英文逗号结束
  eauthor = \secretize{Xincao~Xu},	% 姓名拼音,~代表不会断行的空格
  studentid = \secretize{20128888},	% 仅本科生,学号
  csupervisor = \secretize{刘~~~~凯~~~~教授},	% 导师的姓名
  esupervisor = \secretize{{Prof.~Kai Liu}},	% 导师的姓名拼音
  cassistsupervisor = \secretize{}, % 本科生可选,助理指导教师姓名,不用时请留空为{}
  cextrasupervisor = \secretize{}, % 本科生可选,校外指导教师姓名,不用时请留空为{}
  eassistsupervisor = \secretize{}, % 本科生可选,助理指导教师或/和校外指导教师姓名拼音,不用时请留空为{}
  cpsupervisor = \secretize{丁小明~~工程师}, % 仅专硕,兼职导师姓名
  epsupervisor = \secretize{Eng.~Xiaoming~Ding},	% 仅专硕,兼职导师姓名拼音
  cclass = \rmfamily{2023}年\rmfamily{6}月,	% 博士生和学硕填学科门类,学硕填学科类型
  research_direction = \zihao{3}{车联网},
  edgree = {Degree of Master of Enginnering},	% 专硕填Professional Degree,其他按实情填写
% 提示:如果内容太长,可以用\zihao{}命令控制字号,作用范围:{}内
  cmajor = 工~~~~学,	% 专硕不需填,填写专业名称
  emajor = Material Science and Engineering, % % 专硕不需填,填写专业英文名称
  cmajora = \zihao{3}{计算机科学与技术},
  cmajorb = \zihao{3}{车联网},
  cmajorc = \secretize{雒江涛~~~~教授},
%  cmajord = 2023年6月,
% ===================
% 底部的学院名称和日期
% ===================
  cdepartment = 材料科学与工程学院,	%学院名称
  edepartment = College of Material Science and Engineering,	%学院英文名称
% ===================
% 封面的日期可以自动生成(注释掉时),也可以解除注释手动指定,例如:二〇一六年五月
% ===================
%	mycdate = {2023年6月},
%	myedate = {June 2023},
}% End of \cqusetup
% ===================
%
% 论文的摘要
%
% ===================
\begin{cabstract}	% 中文摘要
现实世界中的众多领域时常会面临各种复杂场景的决策问题,因此实现相关决策的最优化一直是科学研究的热点。然而,随着科学技术的快速发展,大量最优化问题的规模呈爆炸式增长。大规模的优化问题如空中交通流优化、QoS-aware服务优化、大规模交通网络中的车辆路径优化等,通常具有成千上万,甚至达到十万、百万个决策变量,决策变量数量的快速增加,则会导致搜索空间的大小呈现指数增长,这造成启发式搜索算法性能快速恶化,进而造成“维数灾难”。此外,大规模优化问题多具有多峰性、高非线性、不可导性等特性,这同样易引起多峰函数的局部最优解的个数呈指数增长;而当计算资源有限时,算法易陷入局部最优而难以获得优化问题的全局最优解,这将给确定性优化算法带来极大的求解难度。为解决上述问题,已有部分研究通过采用简化搜索空间的方法来求解大规模优化问题,如基于分解的方法和基于降维的方法。尽管这类方法在解决大规模优化问题上已取得了相对优越的性能,但其仍存在显著的不足。其中,基于分解的方法,由于其严重依赖于决策变量之间交互的准确性检测,当其面临复杂的或不可分解的大规模优化问题时,易出现无法有效地分解大规模搜索空间而求解失败的问题;而基于降维的方法,在简化空间中则难以保证全局最优解和高质量解的存在。基于上述求解方法的缺点,本文以求解大规模优化问题中简化搜索空间的方法作为切入点,展开多空间进化搜索算法的设计与研究,主要研究内容包含以下几方面:

\circled{1} 针对大规模连续优化问题,现有简化搜索空间的求解方法,常需考虑决策变量之间关系的假设和要求,因而无法保证全局最优解或高质量解的存在的问题。基于此,设计了一种面向大规模连续优化问题的多空间进化搜索算法。在该算法中,给定一个原问题,除原始问题空间外,构建给定问题的多个简化问题空间,继而学习空间之间的映射。通过映射实现两个空间之间的知识迁移,将在简化问题空间发现的有用的信息迁移给原始问题空间,加速原始问题空间的搜索;同时,将在原始问题空间发现的高质量的解迁移给简化问题空间并引导简化问题空间的搜索方向朝向理想区域演化。此外,为探索不同辅助任务的有效性,对简化问题空间进行阶段性的重构。在CEC2013大规模连续基准测试集的15个问题上对该算法的有效性能进行验证,结果显示,该研究提出的多空间进化搜索(MSES)算法在解决大规模优化问题上,相较于基于协同进化算法DG2、RDG、RDG3,分别在13、13、12个问题上获得了较好的平均目标值;与基于非分解的优化算法DLLSO和MeMAO算法相比,分别在9、13个问题上获得了较好的平均目标值。

\circled{2} 针对大规模离散优化问题,该研究以经典的且具有代表性的车辆路径规划问题为实例,针对难以确定哪种形式对求解车辆路径问题是最有效的形式的问题,设计了多空间模因搜索算法。该算法构建了给定车辆路径问题的不同的简化形式,将简化形式的车辆路径问题作为原始车辆路径问题的辅助任务。在不同形式的车辆路径问题上同时执行模因搜索,并将简化形式的车辆路径问题搜索到的好的解迁移给原始的车辆路径问题,以加速其收敛。该算法在大规模车辆路径问题的基准测试集的65个实例上的实验结果表明,与所用的基线算法和随机算法相比,该研究提出的多空间模因搜索算法在65个实例上均获得了较好的平均开销。

\circled{3} 针对多空间进化搜索算法中资源分配不合理的问题,提出了基于动态资源分配策略的多空间进化搜索算法。该算法设计了一种资源分配合理性检测机制,该机制用于检测不同空间之间资源分配的合理性;同时,构建了一种基于显式贡献和隐式贡献的在线资源统计方法,其中,在不同空间上执行独立进化搜索并优化出最优解的适应度值的提升量定义为显式贡献,而问题空间中迁移解的个体的存活率定义为隐式贡献。并采用参数自适应方法来协调空间的显式贡献和隐式贡献,以进行资源的合理分配,进而提升多空间进化搜索算法的性能。在CEC2013大规模连续基准测试集的15个问题上对该算法的有效性进行验证,结果显示,该研究提出的基于动态资源分配策略的多空间进化搜索(MSES-DRA)算法在解决大规模优化问题上,相较于在分别采用不同优化器DLLSO、SHADE、SaNSDE时的多空间进化搜索(MSES)算法,分别在10、9、9个问题上获得较好的平均目标值;与FCRACC和CCFR2方法相比,分别在10、10个问题上获得较好的平均目标值。
\end{cabstract}
% 中文关键词,请使用英文逗号分隔:
\ckeywords{重庆大学,\LaTeX,\LaTeXe,论文,模板}

\begin{eabstract}	% 英文摘要
Decision problems in complex scenarios widely exist in many fields in real world. The optimization of relevant decisions has drawn much attentation in scientific research. In recent years, with the development of the science and technology, the scale of these optimization problems sharply increases. Generally, large-scale optimization problems have thousands of decision variables, particular examples include optimization of air traffic flow, QoS-aware service, vehicle route optimization in large-scale transportation network, etc. In such scenario, the fast-growing number of decision variables may lead to the exponential growth of search space, which results in deteriorating rapidly on the performance of heuristic search algorithm and “curse of dimension”. Furthermore, the large-scale optimization problems typically involve the characteristics of multimodality, nonlinearity and nondifferentiability, which leads to the soaring number of local optimal solutions of multimodal function increasing exponentially. As a result, it is difficult to find the optimal solution within the limited computing resources, and the algorithm may fall into the local optimization, which has great difficulty and challenge for deterministic optimization algorithm. To address these issues, many approaches to simplify the search space are proposed, which can be categorized into two groups: decomposition-based methods and dimension-reduction-based methods. However, it is worth noting that decomposition-based approaches mostly rely on the accurate detection of the relationships between decision variables, which may fail in solving large-scale optimization problems that possess complex variable interactions, or even nondecomposable. On the other hand, dimension-reduction-methods may lose important optimization information, and it is difficult to guarantees the global optimum or high-quality solutions in reduced solution space. Keeping this in mind, in this thesis, we focus on simplifying the search space of large-scale optimization problem to develop multi-space evolutionary search paradigm. Specifically, main contributions of this thesis can be summarized as follows:
\circled{1} Existing simplifying search space methods for solving large-scale continuous optimization problem often need to consider the problem characterstics. There are assumptions and requirements on the relationships between decision variables of the given optimization problem, and it cannot guarantee the existence of global optimum or high-quality solutions in the search space. To tackle such problem, a new evolutionary search paradigm, namely the multi-space evolutionary search, for solving large-scale continuous optimization problem is proposed. Specifically, given a large-scale optimization problem, besides the original problem space, multiple simplified solution spaces are derived. Moreover, the mapping between these problem spaces is learned, which will be used for knowledge transfer across spaces during the evolutionary search process. In this way, the useful traits found in the simplified problem space can be leveraged to facilitate the search in the original space, while the high-quality solutions found in the original problem space may also guide the search direction in the simplified problem space toward promising areas. Furthermore, to explore the usefulness of diverse auxiliary tasks, the simplified problem space will be re-constructed periodically using the solutions found during the evolutionary search process. The experimental results on 15 functions of CEC2013 large-scale benchmark suite show that the proposed MSES achieved significantly better averaged objective values on 13, 13 and 12 problems in contrast to DG2, RDG, RDG3. Compared with the DLLSO and MeMAO, the proposed MSES obtained significantly better averaged objective values on 9 and 13 problems.
\circled{2} For solving large-scale discrete optimization problems, the multi-space memetic search algorithm is presented by taking the classical vehicle routing problem as an example. Specifically, multiple simplified vehicle routing problems are firstly constructed and acted as the auxiliary tasks of the original vehicle routing problem. The memetic evolutionary search is thus simultaneously performed on simplified vehicle routing problems and original vehicle routing problem. In this way, the useful traits found in the simplified vehicle routing problems can be transferred to original vehicle routing problem to enhance the evolutionary search. The experimental results on 65 instances of commonly used CVRP benchmark suit demonstrate that the proposed method can obtain superior results than the baseline algorithm and random method.
\circled{3} To handle the unreasonable resource allocation problem in multi-space evolutionary search, the multi-space evolutionary search with dynamic resource allocation strategy (MSES-DRA) is proposed, which is designed based on explicit-implicit contributions of spaces according to the interaction between optimal individual and population. In particular, a detection mechanism is presented to measure the reasonableness of assignment in terms of computation resource of spaces. Meanwhile, an online resource statistics based on implicit and explicit contribution is proposed. Specifically, the explicit contributions can be defined by the fitness improvements of the best solutions found by independent evolutionary search performed on different spaces. The implicit contributions can be defined by the survivals of individuals that are transferred solutions across problem spaces. Further, an adaptive technology based on feedback is conducted to balance the assignment in terms of computational resource based on explicit contribution and implicit contribution. The experimental results on 15 problems of CEC2013 large-scale benchmark suite demonstrate that the proposed MSES-DRA method achieved significantly better averaged objective values on 10, 9 and 9 problems in contrast to the MSES method using DLLSO, SHADE and SaNSDE as the optimizers, respectively. Compared with the FCRACC and CCFR2, the proposed MSES-DRA method obtained significantly better averaged objective values on 10 and 10 problems, respectively.  
\end{eabstract}
% 英文关键词,请使用英文逗号分隔,关键词内可以空格:
\ekeywords{Large-scale optimization, Evolutionary multitasking, Knowledge transfer, Dynamic resource allocation, Vehicle routing problem
}

% 封面和摘要配置完成
